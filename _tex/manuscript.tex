% Options for packages loaded elsewhere
\PassOptionsToPackage{unicode}{hyperref}
\PassOptionsToPackage{hyphens}{url}
\PassOptionsToPackage{dvipsnames,svgnames,x11names}{xcolor}
%
\documentclass[
  sn-mathphys-num,
]{sn-jnl}

\usepackage{amsmath,amssymb}
\usepackage{iftex}
\ifPDFTeX
  \usepackage[T1]{fontenc}
  \usepackage[utf8]{inputenc}
  \usepackage{textcomp} % provide euro and other symbols
\else % if luatex or xetex
  \usepackage{unicode-math}
  \defaultfontfeatures{Scale=MatchLowercase}
  \defaultfontfeatures[\rmfamily]{Ligatures=TeX,Scale=1}
\fi
\usepackage{lmodern}
\ifPDFTeX\else  
    % xetex/luatex font selection
\fi
% Use upquote if available, for straight quotes in verbatim environments
\IfFileExists{upquote.sty}{\usepackage{upquote}}{}
\IfFileExists{microtype.sty}{% use microtype if available
  \usepackage[]{microtype}
  \UseMicrotypeSet[protrusion]{basicmath} % disable protrusion for tt fonts
}{}
\makeatletter
\@ifundefined{KOMAClassName}{% if non-KOMA class
  \IfFileExists{parskip.sty}{%
    \usepackage{parskip}
  }{% else
    \setlength{\parindent}{0pt}
    \setlength{\parskip}{6pt plus 2pt minus 1pt}}
}{% if KOMA class
  \KOMAoptions{parskip=half}}
\makeatother
\usepackage{xcolor}
\setlength{\emergencystretch}{3em} % prevent overfull lines
\setcounter{secnumdepth}{-\maxdimen} % remove section numbering
% Make \paragraph and \subparagraph free-standing
\makeatletter
\ifx\paragraph\undefined\else
  \let\oldparagraph\paragraph
  \renewcommand{\paragraph}{
    \@ifstar
      \xxxParagraphStar
      \xxxParagraphNoStar
  }
  \newcommand{\xxxParagraphStar}[1]{\oldparagraph*{#1}\mbox{}}
  \newcommand{\xxxParagraphNoStar}[1]{\oldparagraph{#1}\mbox{}}
\fi
\ifx\subparagraph\undefined\else
  \let\oldsubparagraph\subparagraph
  \renewcommand{\subparagraph}{
    \@ifstar
      \xxxSubParagraphStar
      \xxxSubParagraphNoStar
  }
  \newcommand{\xxxSubParagraphStar}[1]{\oldsubparagraph*{#1}\mbox{}}
  \newcommand{\xxxSubParagraphNoStar}[1]{\oldsubparagraph{#1}\mbox{}}
\fi
\makeatother


\providecommand{\tightlist}{%
  \setlength{\itemsep}{0pt}\setlength{\parskip}{0pt}}\usepackage{longtable,booktabs,array}
\usepackage{calc} % for calculating minipage widths
% Correct order of tables after \paragraph or \subparagraph
\usepackage{etoolbox}
\makeatletter
\patchcmd\longtable{\par}{\if@noskipsec\mbox{}\fi\par}{}{}
\makeatother
% Allow footnotes in longtable head/foot
\IfFileExists{footnotehyper.sty}{\usepackage{footnotehyper}}{\usepackage{footnote}}
\makesavenoteenv{longtable}
\usepackage{graphicx}
\makeatletter
\def\maxwidth{\ifdim\Gin@nat@width>\linewidth\linewidth\else\Gin@nat@width\fi}
\def\maxheight{\ifdim\Gin@nat@height>\textheight\textheight\else\Gin@nat@height\fi}
\makeatother
% Scale images if necessary, so that they will not overflow the page
% margins by default, and it is still possible to overwrite the defaults
% using explicit options in \includegraphics[width, height, ...]{}
\setkeys{Gin}{width=\maxwidth,height=\maxheight,keepaspectratio}
% Set default figure placement to htbp
\makeatletter
\def\fps@figure{htbp}
\makeatother

%%%% Standard Packages

\usepackage{graphicx}%
\usepackage{multirow}%
\usepackage{amsmath,amssymb,amsfonts}%
\usepackage{amsthm}%
\usepackage{mathrsfs}%
\usepackage[title]{appendix}%
\usepackage{xcolor}%
\usepackage{textcomp}%
\usepackage{manyfoot}%
\usepackage{booktabs}%
\usepackage{algorithm}%
\usepackage{algorithmicx}%
\usepackage{algpseudocode}%
\usepackage{listings}%

%%%%

\raggedbottom
\usepackage{multirow}
\usepackage{centernot}
\makeatletter
\@ifpackageloaded{caption}{}{\usepackage{caption}}
\AtBeginDocument{%
\ifdefined\contentsname
  \renewcommand*\contentsname{Table of contents}
\else
  \newcommand\contentsname{Table of contents}
\fi
\ifdefined\listfigurename
  \renewcommand*\listfigurename{List of Figures}
\else
  \newcommand\listfigurename{List of Figures}
\fi
\ifdefined\listtablename
  \renewcommand*\listtablename{List of Tables}
\else
  \newcommand\listtablename{List of Tables}
\fi
\ifdefined\figurename
  \renewcommand*\figurename{Figure}
\else
  \newcommand\figurename{Figure}
\fi
\ifdefined\tablename
  \renewcommand*\tablename{Table}
\else
  \newcommand\tablename{Table}
\fi
}
\@ifpackageloaded{float}{}{\usepackage{float}}
\floatstyle{ruled}
\@ifundefined{c@chapter}{\newfloat{codelisting}{h}{lop}}{\newfloat{codelisting}{h}{lop}[chapter]}
\floatname{codelisting}{Listing}
\newcommand*\listoflistings{\listof{codelisting}{List of Listings}}
\makeatother
\makeatletter
\makeatother
\makeatletter
\@ifpackageloaded{caption}{}{\usepackage{caption}}
\@ifpackageloaded{subcaption}{}{\usepackage{subcaption}}
\makeatother
\ifLuaTeX
  \usepackage{selnolig}  % disable illegal ligatures
\fi
\usepackage[]{natbib}
\bibliographystyle{plainnat}
\usepackage{bookmark}

\IfFileExists{xurl.sty}{\usepackage{xurl}}{} % add URL line breaks if available
\urlstyle{same} % disable monospaced font for URLs
\hypersetup{
  pdftitle={Probabilistic Model of Bilateral Lymphatic Spread in Head and Neck Cancer},
  colorlinks=true,
  linkcolor={blue},
  filecolor={Maroon},
  citecolor={Blue},
  urlcolor={Blue},
  pdfcreator={LaTeX via pandoc}}

\title[Probabilistic Model of Bilateral Lymphatic Spread in Head and
Neck Cancer]{Probabilistic Model of Bilateral Lymphatic Spread in Head
and Neck Cancer}

% author setup
\author*[1,2]{\fnm{Roman} \sur{Ludwig}}\email{roman.ludwig@usz.ch}\author[1,2]{\fnm{Yoel Perez} \sur{Haas}}\email{yoel.perezhaas@usz.ch}\author[1,2]{\fnm{Jan} \sur{Unkelbach}}\email{jan.unkelbach@usz.ch}
% affil setup
\affil[1]{\orgdiv{Department of Physics}, \orgname{University of
Zurich}}
\affil[2]{\orgdiv{Radiation Oncology}, \orgname{University Hospital
Zurich}}

% abstract 


% keywords

\begin{document}
\maketitle

\textsubscript{Source:
\href{https://rmnldwg.github.io/bilateral-paper/manuscript-preview.html}{Article
Notebook}}

\textsubscript{Source:
\href{https://rmnldwg.github.io/bilateral-paper/manuscript-preview.html}{Article
Notebook}}

\section{Introduction}\label{introduction}

\begin{itemize}
\tightlist
\item
  head and neck cancer spreads through the lymphatic network
\item
  may sometimes spread to contralateral side
\item
  spreads more frequently and to larger extent contralaterally when
  tumor extends the mid-sagittal line
\item
  we describe a model based on previously published hidden Markov model
\item
  we extend it to cover the contralateral side, too
\item
  naive approach: make two independent models for each side, but that is
  not supported by the data
\item
  ipsi- and contralateral side are correlated via time of diagnosis,
  which is correlated with T-category
\item
  tumor extension over mid-sagittal line is modelled as random variable
\item
  spread probabilities from tumor to contralateral LNLs in case of
  midline extesnion are linear combinations of these probabilities in
  case of ipsilateral spread and contralateral spread when tumor is
  clearly lateralized
\end{itemize}

\section{Data on Lymphatic Progression
Patterns}\label{data-on-lymphatic-progression-patterns}

We have collected a detailed dataset of patients with newly diagnosed
oropharyngeal squamous cell carcinomas. It reports the involvement of
every patient individually and per lymph node level in tabular form, in
addition to other clinico-pathological information such as age,
T-category, and HPV p16 status.

Their patient records have been collected at four different institutions
and a brief overview over some of their patients' characteristics are
shown in table~\ref{tbl-data-overview}. Note that the data from the
Inselspital Bern and the Centre Léon Bérard only consists of patients
treated with some form of neck dissection. Since this treatment is more
commonly chosen for early T-category patients, they also make up a
larger portion of the respective dataset.

\textsubscript{Source:
\href{https://rmnldwg.github.io/bilateral-paper/manuscript-preview.html}{Article
Notebook}}

\begin{longtable}[]{@{}lllllll@{}}

\caption{\label{tbl-data-overview}Overview over the five datasets from
four different institutions used to train and evaluate our model. Here,
we briefly characterize the total number of OPSCC patients from the
respective institution, their median age, what proportion received some
form of neck dissection, the N0 portion of patients, what percentage
presented with early T-category, and the prevalence of primary tumor
midline extension. For a much more detailed look at the data, visit
\href{https://lyprox.org}{lyprox.org}.}

\tabularnewline

\caption{}\label{T_42bd2}\tabularnewline
\toprule\noalign{}
Institution & Total & Age (median) & Neck Dissection & N0 & Early T-Cat.
& Mid. Ext. \\
\midrule\noalign{}
\endfirsthead
\toprule\noalign{}
Institution & Total & Age (median) & Neck Dissection & N0 & Early T-Cat.
& Mid. Ext. \\
\midrule\noalign{}
\endhead
\bottomrule\noalign{}
\endlastfoot
Centre Léon Bérard & 325 & 60 & 100\% & 19\% & 69\% & 18\% \\
Inselspital Bern & 74 & 61 & 100\% & 18\% & 66\% & 14\% \\
University Hospital Zurich & 287 & 66 & 26\% & 18\% & 52\% & 31\% \\
Vall d\textquotesingle Hebron Barcelona Hospital & 147 & 58 & 5\% & 21\%
& 34\% & 34\% \\

\end{longtable}

\textsubscript{Source:
\href{https://rmnldwg.github.io/bilateral-paper/manuscript-preview.html}{Article
Notebook}}

\subsection{Contralateral Involvement Prevalence}\label{sec-data-strat}

These datasets allow us to investigate correlations between the
involvement of individual LNLs, or between risk factors and patterns of
involvement. In figure~\ref{fig-data-strat}, we have plotted the
prevalence of each contralateral LNL's involvement, stratified by
T-category, ipsilateral number of involved LNLs, and whether the tumor
extended over the mid-sagittal line. A similar but more complete
stratification is also tabulated in the appendix in
table~\ref{tbl-data-strat}.

\begin{figure}

\centering{

\includegraphics{figures/fig-data-strat.png}

}

\caption{\label{fig-data-strat}Contralateral involvement stratified by
T-category (left panel), the number of metastatic LNLs ipsilaterally
(center panel), and whether the primary tumor extended over the
mid-sagittal line or was clearly lateralized (right panel).}

\end{figure}%

The left panel in figure~\ref{fig-data-strat} indicates that T-category
is correlated with contralateral involvement (as it is with overal
involvement). This is simply because T-category may on average be
considered a surrogate for the time between onset of disease and
diagnosis. I.e., a patient with a T4 tumor was -- on average --
diagnosed later than a patient with a T1 tumor. Thus, the former did
have more time to develop metastases.

Similarly, ipsilateral involvement correlates with contralateral
metastasis. The tumor of a patients with many metastases in ipsilateral
LNLs was probably able to spread for longer (or faster) compared to a
tumor in a patient with no nodal disease. This, too, may therefore be
considered a surrogate for the duration of the disease. In rare cases,
bulky nodal disease ipsilaterally may also redirect lymph fluids to the
contralateral side.

Lastly, the right panel in figure~\ref{fig-data-strat} shows that
patients with a tumor crossing the mid-sagittal line show contralateral
involvement vastly more often compared to patients with clearly
lateralized tumors. This makes intutitive sense, because the lymphatic
system in the head and neck region is typically symmetric and thus no
major vessels cross the midline. Therefore, interstitial fluids from the
primary tumor -- which we assume to carry living malignant cells -- may
only reach the blind-ended lymphatic vessels in the contralateral neck
via short-ranged diffusion. Which in turn is only possible when the
primary tumor is close enough to the mid-sagittal line or crosses it.

\subsection{Requirements for a Bilateral Model}\label{sec-requirements}

Based on the observations of the \hyperref[sec-data-strat]{previous
section}, any potential model that aims to also predict the risk for
contralateral nodal involvement, should be able to take the following
into account:

\begin{enumerate}
\def\labelenumi{\arabic{enumi}.}
\tightlist
\item
  More advanced T-category should lead to higher risk for nodal disease.
  One approach to achieve this via the expected time of diagnosis has
  already been developed in the form of a hidden Markov model
  \citep{ludwig_hidden_2021}.
\item
  The degree of ipsilateral involvement should give the model
  information on the time that may have passed between onset and
  diagnosis of the disease. This should come in addition to what can be
  inferred about this time from T-category alone.
\item
  A tumor that extends over the mid-sagittal line should yield
  contralateral metastases with much higher probability.
\end{enumerate}

Over the course of this work, we will first briefly recap the mentioned
HMM in section~\ref{sec-unilateral}, which was so far used to model
ipsilateral lymphatic progression only. Then, we intuitively extend it
to include the contralateral side as well in
section~\ref{sec-ext-to-contra}. In this section, we also introduce a
way of modelling the tumor's midline extension as a random variable
(section~\ref{sec-midline}) and lastly talk about how it may affect the
contralateral spread in section~\ref{sec-params-symmetry}.

\subsection{Data Availability}\label{data-availability}

The entire data, including additional patients with tumors in other
primary locations than the oropharynx, is publicly available: It may be
\href{https://lyprox.org/patients/dataset}{downloaded from LyProX} where
it can be interactively explored too,
\href{https://github.com/rmnldwg/lydata}{from GitHub},
\href{https://zenodo.org/search?q=lydata}{from zenodo}, or via the
\emph{Data-in-Brief} publications \citet{ludwig_dataset_2022} and
\citet{ludwig_multicentric_2023}. Although these publications do not
include the most recent dataset addition from Vall d'Hebron Barcelona
Hospital.

\section{Unilateral Model for Lymphatic
Progression}\label{sec-unilateral}

\begin{figure}

\centering{

\includegraphics[width=0.4\textwidth,height=\textheight]{static/small-graph.png}

}

\caption{\label{fig-small-graph}Directed acyclic graph (DAG)
representing the abstract lymphatic network in the head and neck region.
Blue nodes are the LNLs' hidden random variables, the red node
represents the tumor, and the orange square nodes depict the binary
observed variables. Red and blue arcs symbolize the probability of
lymphatic spread along that edge during one time-step. The orange arcs
represent the sensitivity and specificity of the observational modality
(e.g.~CT, MRI, pathology, \ldots).}

\end{figure}%

Our first model to predict the lymphatic progression of HNSCC was
introduced using Bayesian networks \citep{pouymayou_bayesian_2019}. We
subsequently extended this work to a hidden Markov model (HMM)
\citep{ludwig_hidden_2021} to allow an intuitive inclusion of T-category
into the predictions. We will briefly summarize this HMM's formalism
before building on it to include the contralateral spread in
section~\ref{sec-ext-to-contra}.

We model a patient's state of involvement at an abstract time-step \(t\)
as a vector of hidden binary random variables:

\begin{equation}\phantomsection\label{eq-state-def}{
\mathbf{X}[t] = \begin{pmatrix} X_v[t] \end{pmatrix} \qquad v \in \left\{ 1, 2, \ldots, V \right\}
}\end{equation}

Here, \(V\) is the number of LNLs the model considers. The values a
LNL's hidden binary RV may take on are \(X_v[t] = 0\) (\texttt{False}),
meaning the LNL \(v\) is healthy or free of metastatic disease, or
\(X_v[t] = 1\) (\texttt{True}), corresponding to some for of tumor
presence (i.e., occult or clinical).

Since the state vector \(\mathbf{X}[t]\) is \(V\)-dimensional and
binary, there are \(2^V\) distinct possible lymphatic involvement
patterns, which we enumerate from
\(\boldsymbol{\xi}_0 = \begin{pmatrix} 0 & 0 & \cdots & 0 \end{pmatrix}\)
to
\(\boldsymbol{\xi}_{2^V} = \begin{pmatrix} 1 & 1 & \cdots & 1 \end{pmatrix}\).

Any hidden Markov model is fully described by three quantities:

\begin{enumerate}
\def\labelenumi{\arabic{enumi}.}
\tightlist
\item
  A starting state \(\mathbf{X}[t=0]\) at time \(t=0\) just before the
  patient's tumor formed. In our case, this is always the state where
  all LNLs are still healthy \(\boldsymbol{\xi}_0\).
\item
  The \emph{transition matrix}
  \begin{equation}\phantomsection\label{eq-trans-matrix}{
  \mathbf{A} = \left( A_{ij} \right) = \big( P \left( \mathbf{X}[t+1] = \boldsymbol{\xi}_j \mid \mathbf{X}[t] = \boldsymbol{\xi}_i \right) \big)
  }\end{equation} where the value at row \(i\) and column \(j\)
  represents the probability to transition from state
  \(\boldsymbol{\xi}_i\) to \(\boldsymbol{\xi}_j\) during the time-step
  from \(t\) to \(t+1\). Note that we prohibit self-healing, meaning
  that during a transition, no LNL may change their state from
  \(X_v[t]=1\) to \(X_v[t+1]=0\). This effectively masks large parts of
  the transition matrix to be zero.
\item
  Lastly, the \emph{observation matrix}
  \begin{equation}\phantomsection\label{eq-obs-matrix}{
  \mathbf{B} = \left( B_{ij} \right) = \big( P \left( \mathbf{Z} = \boldsymbol{\zeta}_j \mid \mathbf{X}[t_D] = \boldsymbol{\xi}_i \right) \big)
  }\end{equation} where in row \(i\) and at column \(j\) we find the
  probability to \emph{observe} a lymphatic involvement pattern
  \(\mathbf{Z} = \boldsymbol{\zeta}_j\), given that the true (but
  hidden) state of involvement at the time of diagnosis \(t_D\) is
  \(\mathbf{X}[t_D] = \boldsymbol{\xi}_i\).
\end{enumerate}

Note that the transition matrix \(\mathbf{A}\) is parametrized using the
spread probabilities of a directed acyclic graph (DAG) that we define as
the underlying mechanistic representation of the lymphatic network. An
example of such a DAG is shown in figure~\ref{fig-small-graph}.

Using the introduced quantities, we can evolve the distribution of all
possible hidden states from \(\mathbf{X}[t=0] = \boldsymbol{\xi}_0\)
step by step, by successively multiplying this vector with the
transition matrix \(\mathbf{A}\). At the time of the diagnosis \(t_D\),
we multiply the result with the observation matrix \(\mathbf{B}\). We
may then look up the likelihood of a patient presenting with the
diagnosis \(\mathbf{Z}=\boldsymbol{\zeta}_i\) in the \(i\)-th entry of
the final result.

However, the remaining issue is that the value of \(t_D\) is unknown,
i.e.~over how many time-steps the HMM should be evolved. We solve this
problem by marginalizing over the time of diagnosis. Different
distributions over the diagnosis times can then be choosen based on
T-category. For instance, the mean of the time-prior to marginalize over
the diagnosis time for early T-category patients
\(P\left( t_D \mid \text{early} \right)\) may be shifted towards earlier
times than the one for advanced T-category patients
\(P\left( t_D \mid \text{early} \right)\). This gives us for example

\[
P\left( \mathbf{X} \mid \text{T}x = \text{early} \right) = \sum_{t=0}^{t_\text{max}} P \left( \mathbf{X} \mid t \right) \cdot P(t \mid \text{early})
\]

For later use, we define at this point the matrices
\(\boldsymbol{\Lambda}\):

\begin{equation}\phantomsection\label{eq-lambda-matrix}{
\boldsymbol{\Lambda} = P \left( \mathbf{X} \mid \mathbf{t} \right) = \begin{pmatrix}
\boldsymbol{\pi}^\intercal \cdot \mathbf{A}^0 \\
\boldsymbol{\pi}^\intercal \cdot \mathbf{A}^1 \\
\vdots \\
\boldsymbol{\pi}^\intercal \cdot \mathbf{A}^{t_\text{max}} \\
\end{pmatrix}
}\end{equation}

Where the \(k\)-th row in this matrix corresponds to the distribution
over hidden states after \(t=k-1\) time-steps.

In this work, we use binomial distributions
\(\mathfrak{B} \left( t_D, p_{\text{T}x} \right)\) as time-priors which
have one free parameter \(p_{\text{T}x}\) for each group of patients we
differentiate based on T-category. Also, we fix \(t_\text{max} = 10\),
which means that the expected number of time-steps from the onset of a
patient's disease to their diagnosis is
\(\mathbb{E}\left[ t_D \right] = 10 \cdot p_{\text{T}x}\).

\subsection{Likelihood Function}\label{likelihood-function}

With the formalism introduced above, we can write the likelihood
function for a patient to present with a diagnosis consisting of an
observed state and a T-category
\(d = \left( \boldsymbol{\zeta}_i, \text{T}x \right)\) as follows:

\begin{equation}\phantomsection\label{eq-single-patient-llh}{
\ell = P \left( \mathbf{Z} = \boldsymbol{\zeta}_i \mid \text{T}x \right) = \sum_{t=0}^{t_\text{max}} \left[ \boldsymbol{\xi}_0 \cdot \mathbf{A} \cdot \mathbf{B} \right]_i \cdot P \left( t \mid \text{T}x \right)
}\end{equation}

Above, the quantity inside \(\left[ \ldots \right]_i\) denotes the
\(i\)-th component of the vector that is the result of the vector and
matrix multiplications in the square brackets. Note that it is also
possible to account for missing involvement information: If a diagnosis
(like fine needle aspiration (FNA)) is only available for a subset of
all LNLs, we can sum over all those possible complete observed states
\(\boldsymbol{\zeta}_j\) that match the provided diagnosis.

The single-patient likelihood \(\ell\) in
equation~\ref{eq-single-patient-llh} depends on the spread parameters
shown in figure~\ref{fig-small-graph} via the transition matrix
\(\mathbf{A}\) and on the binomial parameters \(p_{\text{T}x}\) via
time-priors. In this work, we will only differentiate between ``early''
(T1 \& T2) and ``advanced'' (T3 \& T4) T-categories. Therefore, our
parameter space is:

\begin{equation}\phantomsection\label{eq-param-space}{
\boldsymbol{\theta} = \left( \left\{ b_v \right\}, \left\{ t_{vr} \right\}, p_\text{early}, p_\text{adv.} \right) \quad \text{with} \quad \genfrac{}{}{0pt}{2}{v\leq V}{r\in\operatorname{pa}(v)}
}\end{equation}

And it is our goal to infer the values of these parameters for a given
dataset \(\mathcal{D} = \left( d_1, d_2, \ldots, d_N \right)\) of OPSCC
patients. The likelihood of these \(N\) diagnoses is simply the product
of their individual likelihoods as defined in
equation~\ref{eq-single-patient-llh}. For numerical reasons, we
typically compute the data likelihood in log space:

\begin{equation}\phantomsection\label{eq-log-likelihood}{
\log \mathcal{L} \left( \mathcal{D} \mid \boldsymbol{\theta} \right) = \sum_{i=1}^N \log \ell_i
}\end{equation}

The methodology we use to infer the model's parameters is detailed in
section~\ref{sec-sampling}.

\subsection{Model Prediction in the Bayesian
Context}\label{model-prediction-in-the-bayesian-context}

Our stated goal is to compute the risk for a patient's true nodal
involvement state \(\mathbf{X}\), \emph{given} their individual
diagnosis \(d = \left( \boldsymbol{\zeta}_k, \text{T}x \right)\). Using
Bayes' law, this is written as:

\begin{equation}\phantomsection\label{eq-uni-bayes-law}{
P \big( \mathbf{X} \mid \mathbf{Z}=\boldsymbol{\zeta}_k, \boldsymbol{\hat{\theta}}, \text{T}x \big) = \frac{P \left( \boldsymbol{\zeta}_k \mid \mathbf{X} \right) P \big( \mathbf{X} \mid \boldsymbol{\hat{\theta}}, \text{T}x \big)}{\sum_{i=0}^{2^V} P \left( \boldsymbol{\zeta}_k \mid \mathbf{X}=\boldsymbol{\xi}_i \right) P \big( \mathbf{X}=\boldsymbol{\xi}_i \mid \boldsymbol{\hat{\theta}}, \text{T}x \big)}
}\end{equation}

The term \(P \left( \boldsymbol{\zeta}_k \mid \mathbf{X} \right)\) is
defined solely by sensitivity and specificity of the diagnostic
modality. Terms like this already appeared in the definition of the
observation matrx in equation~\ref{eq-obs-matrix}. The \emph{prior}
\(P \big( \mathbf{X} \mid \boldsymbol{\hat{\theta}} \big)\) in the above
equation is the crucial term that is supplied by a trained model and its
parameters \(\boldsymbol{\hat{\theta}}\).

It is possible to compute this \emph{posterior} probability of true
involvement not only for one fully defined state \(\mathbf{X}\), but
also for e.g.~individual LNLs: For example, the risk for involvement in
level IV would be a marginalization over all states
\(\boldsymbol{\xi}_i\), where \(\xi_{i4}=1\). Formally:

\begin{equation}\phantomsection\label{eq-marg-over-posterior}{
P \big( \text{IV} \mid \mathbf{Z}=\boldsymbol{\zeta}_k, \boldsymbol{\hat{\theta}}, \text{T}x  \big) = \sum_{k \, : \, \xi_{k4}=1} P \big( \mathbf{X} = \boldsymbol{\xi}_k \mid \boldsymbol{\zeta}_k, \boldsymbol{\hat{\theta}}, \text{T}x  \big)
}\end{equation}

\section{Extension to a Bilateral Model}\label{sec-ext-to-contra}

A naive approach to model the contralateral lymphatic spread would be to
simply employ two independent unilateral models as introduced in
section~\ref{sec-unilateral}. During training, one could even enforce
some shared parameters between these two models, like the
parameterization of the distributions over diagnose times or the spread
among the LNLs. Additionally, we could think of an approach to
incorporate the primary tumor's mid-sagittal extension as a risk factor.

However, this approach lacks a way to describe the correlation between
ipsi- and contralateral involvement. This is displayed in
table~\ref{tbl-data-strat} and shows how often the contralateral LNLs I,
II, III, and IV were involved, given all possible combinations of
midline extension, T-category, and ipsilateral LNL III involvement.
Unsurprisingly, the prevalence for contralateral involvement is
consistently higher when the tumor extends over the mid-sagittal line or
is of later T-category. But it is also more frequent when the
ipsilateral side shows more severe involvement, which is here shown via
the surrogate LNL III.

Thus, we attempt to extend the formalism in section~\ref{sec-unilateral}
in such a way that the model's ipsi- and contralateral side evolve
synchronously. To achieve that, we start by writing down the posterior
distribution of involvement an analogy to
equation~\ref{eq-uni-bayes-law}, which is now a joint probability of an
involvement \(\mathbf{X}^\text{i}\) ipsilaterally \emph{and} an
involvement \(\mathbf{X}^\text{c}\) contralaterally, given a diagnosis
of the ipsilateral LNLs \(\mathbf{Z}^\text{i}\) and of the contralateral
ones \(\mathbf{Z}^\text{c}\):

\begin{equation}\phantomsection\label{eq-bilateral-bayes}{
P \left( \mathbf{X}^\text{i}, \mathbf{X}^\text{c} \mid \mathbf{Z}^\text{i}, \mathbf{Z}^\text{c} \right) = \frac{P \left( \mathbf{Z}^\text{i}, \mathbf{Z}^\text{c} \mid \mathbf{X}^\text{i}, \mathbf{X}^\text{c} \right) P \left( \mathbf{X}^\text{i}, \mathbf{X}^\text{c} \right)}{P \left( \mathbf{Z}^\text{i}, \mathbf{Z}^\text{c} \right)}
}\end{equation}

For the sake of brevity, we omit the dependency on the parameters and
the T-category here.

The likelihood of the diagnoses given a hidden state simply factorise:
\(P \left( \mathbf{Z}^\text{i}, \mathbf{Z}^\text{c} \mid \mathbf{X}^\text{i}, \mathbf{X}^\text{c} \right) = P \left( \mathbf{Z}^\text{i} \mid \mathbf{X}^\text{i} \right) \cdot P \left( \mathbf{Z}^\text{c} \mid \mathbf{X}^\text{c} \right)\).
And the two factors are contained in the observation matrices
\(\mathbf{B}^\text{i}\) and \(\mathbf{B}^\text{c}\).

The term representing the model's prior probability of hidden
involvement does not factorize. However, if we assume that lymphatic
spread typically does not cross the mid-sagittal line, we can write it
as a factorising sum:

\begin{equation}\phantomsection\label{eq-bilateral-marginal}{
\begin{aligned}
P \left( \mathbf{X}^\text{i}, \mathbf{X}^\text{c} \right) &= \sum_{t=0}^{t_\text{max}} P(t) \cdot P \left( \mathbf{X}^\text{i}, \mathbf{X}^\text{c} \mid t \right) \\
&= \sum_{t=0}^{t_\text{max}} P(t) \cdot P \left( \mathbf{X}^\text{i} \mid t \right) \cdot P \left( \mathbf{X}^\text{c} \mid t \right)
\end{aligned}
}\end{equation}

This assumption makes intuitive sense: The two sides of the lymphatic
network in a typical patient are approximately mirror images of each
other. Thus, no major vessels cross the mid-sagittal line. There may,
however, be diffusion of lymph fluid accross this line or bulky
involvement that redirects lymphatic drainage significantly.

Using this assumption along with equation~\ref{eq-lambda-matrix}, we can
write the above distribution algebraically as a product:

\begin{equation}\phantomsection\label{eq-bilateral-marginal-algebra}{
P \left( \mathbf{X}^\text{i} = \boldsymbol{\xi}_n, \mathbf{X}^\text{c} = \boldsymbol{\xi}_m \right) = \left[ \boldsymbol{\Lambda}^\intercal_\text{i} \cdot \operatorname{diag} P(\mathbf{t}) \cdot \boldsymbol{\Lambda}_\text{c} \right]_{n,m}
}\end{equation}

\subsection{Modelling Midline Extension}\label{sec-midline}

To account for the increased prevalence of involvement on the
contralateral side when the tumor is not clearly lateralized anymore, we
also model the tumor's extension over the mid-sagittal line as a binary
random variable. It starts lateralized and at every time-step there is a
finite probability \(p_\epsilon\) that the tumor grows over the symmetry
plane of the patient.

Technically, the introduction of this additional random variable doubles
the space of the hidden states and therefore quadruples the size of the
transition matrix \(\mathbf{A}\). However, since we assume no
correlation between the tumor's lateralization and the metastases in the
LNLs, we can evolve the two parts separately.

The probabilities to find a patient with a clearly lateralized tumor or
one that extends over the mid-sagittal line after \(t\) time-steps are
then given by

\[
\begin{aligned}
P(\epsilon = \texttt{False} \mid t) &= (1 - p_\epsilon)^t \\
P(\epsilon = \texttt{True} \mid t) &= 1 - P(\epsilon = \texttt{False} \mid t)
\end{aligned}
\]

In figure~\ref{fig-model-midext-evo}, we visualize how the prior
distribution over diagnose times \(P(t)\), the conditional probability
of midline extension \(P(\epsilon \mid t)\), and their joint
\(P(\epsilon, t)\) evolve over the course of a patient evolution.

\begin{figure}

\centering{

\includegraphics[width=0.5\textwidth,height=\textheight]{figures/fig-model-midext-evo.png}

}

\caption{\label{fig-model-midext-evo}The top panel shows the prior
probability to get diagnosed at time-step \(t\) for early and late
T-category tumors as bars. Also in the top panel, we plot the
conditional probability of the tumor's midline extension
(\(\epsilon=\texttt{True}\)), given the time-step \(t\) as a line plot.
In the bottom panel, we show the joint probability of getting diagnosed
in time-step \(t\) \emph{and} having a tumor that crosses the midline.}

\end{figure}%

To get the joint probability over the ipsi- and contralateral hidden
states, as well as the state of the tumor's midline extension
\(P \left( \mathbf{X}^\text{i}, \mathbf{X}^\text{c}, \epsilon \right)\),
we simply add the above terms to the marginalization in
equation~\ref{eq-bilateral-marginal}:

\[
P \left( \mathbf{X}^\text{i}, \mathbf{X}^\text{c}, \epsilon \right) = \sum_{t=0}^{t_\text{max}} P(t) \cdot P(\epsilon \mid t) \cdot P \left( \mathbf{X}^\text{i} \mid t \right) \cdot P \left( \mathbf{X}^\text{c} \mid t \right)
\]

Again, this can be written algebraically, by defining the vector
\(P(\mathbf{t}, \epsilon) = P(\mathbf{t}) \cdot P(\epsilon \mid \mathbf{t})\),
to achieve the same form as in
equation~\ref{eq-bilateral-marginal-algebra}:

\[
P \left( \mathbf{X}^\text{i} = \boldsymbol{\xi}_n, \mathbf{X}^\text{c} = \boldsymbol{\xi}_m, \epsilon \right) = \left[ \boldsymbol{\Lambda}^\intercal_\text{i} \cdot \operatorname{diag} P(\mathbf{t}, \epsilon) \cdot \boldsymbol{\Lambda}_\text{c} \right]_{n,m}
\]

In figure~\ref{fig-model-state-dist} we plot the state distribution for
a full midline model as two separate heatmaps. To keep it readable, this
example model only considers the LNLs II, III, and IV ipsi- and
contralaterally.

\begin{figure}

\centering{

\includegraphics{figures/fig-model-state-dist.png}

}

\caption{\label{fig-model-state-dist}3D state distribution of a midline
model with 3 LNLs (II, III, and IV) in both sides of the neck.}

\end{figure}%

\subsection{Parameter Symmetries}\label{sec-params-symmetry}

In general, the matrices \(\boldsymbol{\Lambda}_\text{i}\) and
\(\boldsymbol{\Lambda}_\text{c}\) could be parameterized using a
disjoint set of parameters. I.e., the ipsi- and contralateral spread
rates are entirely different. However, using two sensible assumptions,
we can reduce the parameter space by sharing some parameters between the
sides:

\begin{enumerate}
\def\labelenumi{\arabic{enumi}.}
\tightlist
\item
  We assume the spread \emph{among} the LNLs to be same on both sides.
  It is reasonable to assume the lymphatic system is symmetric. Thus,
  the spread rates from one LNL to the other should be symmetric, too.
  Formally, this means\\
  \begin{equation}\phantomsection\label{eq-symmetries}{
  \begin{aligned}
  b_v^\text{c} &\neq b_v^\text{i} \\
  t_{rv}^\text{c} &= t_{rv}^\text{i}
  \end{aligned}
  }\end{equation} for all \(v \leq V\) and
  \(r \in \operatorname{pa}(v)\).
\item
  The tumor's spread to the contralateral side in case of an extension
  over the midline is larger than if it was clearly lateralized, but
  smaller than its spread to the ipsilateral side. This assumption stems
  from a simple thought experiment: Consider moving the tumor from a
  clearly lateralized position accross the mid-sagittal plane to the
  same position, but on the contralateral side. In the beginning we
  would have \(b_v^\text{c} < b_v^\text{i}\), while in the end, the
  situation is reversed. If a tumor extends over the mid-sagittal line,
  its contralateral spread rate can be expected to be in between these
  two extremes. We encode this in a \emph{mixing parameter} \(\alpha\)
  that captures a ``degree of asymmetry'':\\
  \begin{equation}\phantomsection\label{eq-mixing}{
  b_v^{\text{c},\epsilon=\texttt{True}} = \alpha \cdot b_v^\text{i} + (1 - \alpha) \cdot b_v^{\text{c},\epsilon=\texttt{False}}
  }\end{equation} This means the model now uses three different sets of
  parameters to describe the spread from the tumor to the LNLs:
  \(b^\text{i}_v\) for the spread to the ipsilateral LNLs,
  \(b_v^{\text{c},\epsilon=\texttt{False}}\) for the spread to the
  contralateral LNLs as long as the tumor is clearly lateralized, and
  finally \(b_v^{\text{c},\epsilon=\texttt{True}}\) when it crosses the
  midline. Note, however, that these three sets of spread rates only
  account for \(2 \cdot 2^V + 1\) parameters, since they are coupled via
  the mixing parameter \(\alpha\).
\end{enumerate}

Together with the explicit modelling of the tumor's midline extension
\(\epsilon\) from section~\ref{sec-midline}, we now have a model that
may be capable of capturing the higher prevalence of contralateral
involvment that comes with tumors extending over the mid-sagittal line.

\section{Methods}\label{methods}

In this section, we detail how the experiments were performed. Every
figure, table, and result is fully reproducible via the GitHub
repository
\href{https://github.com/rmnldwg/bilateral-paper}{\texttt{rmnldwg/bilateral-paper}}.
It also contains the raw manuscript and instructions on how to recreate
all figures, tables, and the final document.

\subsection{Involvement Data
Consensus}\label{involvement-data-consensus}

It is possible to provide our model with multiple different diagnostic
modalities, each being characterized by different pairs of sensitivity
and specificity. However, we instead chose to combine them into a single
``consensus'' diagnosis before parameter inference. We opted for this
because the literature values of sensitivity and specificity
\citep{debondt_detection_2007} of imaging modalities like MRI and CT do
not plausibly match some of our observations: In the USZ cohort, 78\% of
OPSCC patients where diagnosed with ipsilateral LNL II involvement via
diagnostic imaging. This is virtually impossible with sensitivities
around 80\% and specificities lower than 100\%.

Our pre-training consensus was formed by considering all reported
diagnostic information for a particular patient and LNL. When conflicts
arose, we computed the \emph{most likely} true state of involvement
using the literature sensitivity and specificity values
\citep{debondt_detection_2007}.

\subsection{MCMC Sampling}\label{sec-sampling}

For parameter inference, we used the Python package
\href{https://emcee.readthedocs.io/en/stable/}{\texttt{emcee}}
\citep{foreman-mackey_emcee_2013}. It implements efficient MCMC sampling
algorithms that employ multiple parallel samplers for affine invariance
and better performance on multi-core CPUs. The
\href{https://emcee.readthedocs.io/en/stable/}{\texttt{emcee}} library
was provided with the likelihood implemented by our
\href{https://lymph-model.readthedocs.io/en/stable/}{\texttt{lymph-model}}
Python package.

For each dimension in the parameter space of the model, we initialized
12 of these parallel samplers, called ``walkers'', with random values in
the unit cube. Every time all of these walkers advanced 50 steps, the
autocorrelation time of the chains was estimated. For short chains, this
estimate is not trustworthy, but stabilizes for longer chains. We
therefore considered a sampling to be converged when two criteria where
met:

\begin{enumerate}
\def\labelenumi{\arabic{enumi}.}
\tightlist
\item
  The change in the autocorrelation time was less then 5.0e-2.
\item
  The estimate of the autocorrelation dropped below \(n\) / 50 where
  \(n\) is the length of the chain up to that point.
\end{enumerate}

All samples up to this convergence - called the \emph{burn-in phase} -
were discarded. We only kept another 10 samples after that, which were
spaced 10 steps apart.

\subsection{Computing the Observed and Predicted Prevalence of
Involvement Patterns}\label{sec-prevalence}

We want to assess the model's capability to approximate the distribution
of lymphatic involvement patterns seen in the data. To that end, we
compare the prevalence of some invovlement patterns under selected
scenarios with the model's prediction for how often these involvements
it expects to see, given these scenarios.

In this context, a ``scenario'' includes the patient's T-category
\(\text{T}x\) and whether the patient's tumor extended over the
mid-sagittal line, i.e.~\(\epsilon=\texttt{True}\) or
\(\epsilon=\texttt{False}\).

An involvement pattern specifies for each ipsi- and contralateral LNL
whether it is ``healthy'', ``involved'', or ``masked''. If it is
``masked'', we essentially state that we are not interested in the
involvement of that LNL and the prevalence will be marginalized over
this LNL's involvement.

For example, we may be interested in the prevalence of contralateral LNL
II involvement (i.e., contra LNL II ``involved'' and all other LNLs
``masked'') under the scenario of early T-category (T0-T2) and no
midline extension (\(\epsilon=\texttt{False}\)). To compute this
prevalence in the data, we select all patients of this scenario (in our
data, this amounts to 379 patients). Of those, 28 were found to harbor
metastases in their contralateral LNL II. Therefore, the prevalence is
7\%.

When displaying this data prevalence, we often choose to draw a
\emph{beta posterior} over the ``true'' prevalence, hinting at the fact
that our data merely represents a limited sample. The beta posterior
follows from a uniform beta distribution as prior and a binomial
likelihood for the number of patients with the involvement of interest,
given the parameer for the ``true'' prevalence. The resulting
distribution has its maximum at the observed prevalence, but in addition
gives a visual intuition for the variance of of the observed quantity.
I.e., when we observe 3 out of 10 events, the beta posterior is much
wider than if we observe 300 out of 1000 for he same prevalence. It also
allows us to check not only if the model is accurate, but also whether
it reflects the uncertainty contained in the data.

Predicting the prevalence using our model amounts to computing the
following probability:

\[
P \left( \text{II}^\text{c} \mid \epsilon=\texttt{False}, \text{T}x=\text{early} \right) = \frac{P \left( \text{II}^\text{c}, \epsilon=\texttt{False} \mid \text{T}x=\text{early} \right)}{P \left( \epsilon=\texttt{False} \mid \text{T}x=\text{early} \right)}
\]

In the enumator, we marginalize over all ipsi- and contralateral LNLs'
involvements, except for LNL II contralaterally. This is similar to the
marginalization in equation~\ref{eq-marg-over-posterior}, although we
are summing over different quantities. In the denominator, we can simply
insert the joint distribution over midline extension and diagnose time
\(P \left( \epsilon, t \right)\) marginalized over \(t\) using the early
T-category's time-prior.

Since we compare it to the data, which does not report true but only
observed involvement -- although pathologically investigated LNLs may be
as close as possible to the ground truth -- we do not consider
posteriors of the form \(P \left( \mathbf{X} \mid \mathbf{Z} \right)\)
here. Instead, we compute probabilities of observed involvement
\(P \left( \mathbf{Z} \right)\), as in the likelihood
equation~\ref{eq-single-patient-llh}.

When plotted, we usually display histograms over the model's
predictions. Each of their values was computed from a different
parameter set drawn during MCMC sampling, effectively giving us a
distribution over the prevalences. Ideally, the histograms approximate
the location and width of the Beta posteriors when attempting to
describe the data they were trained on.

Note that we decided to omit the y-axis ticks and labels in these
figures over prevalences and risks. The y-axis in these plots measures
the probability density and its numerical values are not intuitively
interpretable. Instead, we occasionally use the freed space to label
e.g.~rows of subplots.

\section{Results}\label{results}

First, in figure~\ref{fig-model-burnin-history}, we verify the sampling
converged successfully by inspecting two monitoring quantities: The
autocorrelation time of the MCMC chain and the acceptance fractions of
the parallel walkers.

\begin{figure}

\centering{

\includegraphics{figures/fig-model-burnin-history.png}

}

\caption{\label{fig-model-burnin-history}Monitoring quantities during
the burn-in phase of the parameter sampling. Left: The autocorrelation
time of the sampling chain estimated at different sampling steps. We
consider the chain converged when the estimate of the autocorrelation
time is stable and drops below the trust threshold of \(n/50\) where
\(n\) is the number of steps. Right: Fraction of accepted MCMC proposals
averaged over all parallel walkers. Values around 30\% indicate good
mixing of the walkers.}

\end{figure}%

In table~\ref{tbl-midline-params}, we tabulate the mean and standard
deviation of the sampled parameters for the full midline model.

\textsubscript{Source:
\href{https://rmnldwg.github.io/bilateral-paper/manuscript-preview.html}{Article
Notebook}}

\begin{longtable}[]{@{}lll@{}}

\caption{\label{tbl-midline-params}Mean sampled parameter estimates of
the midline model and the respective standard deviation.}

\tabularnewline

\caption{}\label{T_c1c15}\tabularnewline
\toprule\noalign{}
Parameter & Mean & Std. Dev. \\
\midrule\noalign{}
\endfirsthead
\toprule\noalign{}
Parameter & Mean & Std. Dev. \\
\midrule\noalign{}
\endhead
\bottomrule\noalign{}
\endlastfoot
Mid. ext. probability & 8.13\% & ± 0.60\% \\
ipsi: T ➜ II & 35.45\% & ± 1.75\% \\
ipsi: T ➜ III & 5.59\% & ± 0.82\% \\
ipsi: T ➜ IV & 0.91\% & ± 0.21\% \\
contra: T ➜ II & 2.64\% & ± 0.35\% \\
contra: T ➜ III & 0.16\% & ± 0.09\% \\
contra: T ➜ IV & 0.21\% & ± 0.10\% \\
Mixing ⍺ & 21.01\% & ± 3.24\% \\
II ➜ III & 13.46\% & ± 1.95\% \\
III ➜ IV & 15.74\% & ± 2.24\% \\
late T-cat. binom. prob. & 45.35\% & ± 2.57\% \\

\end{longtable}

\textsubscript{Source:
\href{https://rmnldwg.github.io/bilateral-paper/manuscript-preview.html}{Article
Notebook}}

\subsection{Prevalence Predictions}\label{prevalence-predictions}

We want to investigate whether and to what extent the model can fulfill
the requirements laid out in section~\ref{sec-requirements}. To that
end, we compare its predictions for contralateral involvement against
observations in the data. This is done given scenarios that differ in
T-category and/or midline extension and/or ipsilateral involvement.

\subsubsection{Dependence of Contralateral Involvement on T-Category and
Midline
Extension}\label{dependence-of-contralateral-involvement-on-t-category-and-midline-extension}

In figure~\ref{fig-model-prevalences-overall}, we plot the prevalence of
contralateral involvement of the LNLs II, III, and IV for the four
scenarios made up of the possible combinations of early and late
T-category, as well as lateralized and midline extending tumors.

\begin{figure}

\centering{

\includegraphics{figures/fig-model-prevalences-overall.png}

}

\caption{\label{fig-model-prevalences-overall}Comparison of predicted
(histograms) vs observed (beta posteriors) prevalences. Shown for the
contralateral LNLs II (blue), III (orange), and IV (green). The top row
shows scenarios with early T-category tumors, the bottom row for late
T-category ones. The left column depicts scenarios where the primary
tumor is clearly lateralized, the right column scenarios of tumors
extending over the mid-sagittal line. This figure illustrates the
model's ability to describe the prevalence of different combinations of
scenarios involving the risk factors T-category and midline extension.}

\end{figure}%

Figure~\ref{fig-model-prevalences-overall} shows nicely how the model is
capable of accurately taking the most important risk factors,
i.e.~T-category and midline extension, into account. As observed in the
data, the model predicts that the prevalence of contralateral LNL II
involvement jumps from below 8\% for early T-category lateralized tumors
to almost 40\% when the tumor is of advanced T-category and crosses the
mid-sagittal line.

However, for early T-category scenarios with midline extension, the
model does seem to overestimate contralateral LNL II and III
invovlement. This likely stems from the the small sample size of this
relatively rare scenario as hinted at by the wide beta posteriors.

\subsubsection{Correlation between Ipsi- and Contralateral
Involvement}\label{correlation-between-ipsi--and-contralateral-involvement}

\begin{figure}

\centering{

\includegraphics{figures/fig-model-prevalences-with-ipsi.png}

}

\caption{\label{fig-model-prevalences-with-ipsi}Comparison of predicted
(histograms) vs observed (beta posteriors) prevalences. Shown are four
scenarios, all including contralateral LNL II involvement: Early
T-category and an ipsilateral N0 neck (green), early T-category and
ipsilateral LNL II involvement (blue), as well as the same two scenarios
but for advanced T-category (orange and red). The figure shows that the
model is capable of describing the correlation between ipsi- and
contralateral involvement. Although for the scenario of LNL II
involvement in both sides, the prediction's split between early and
advanced T-category is not large enough.}

\end{figure}%

In figure~\ref{fig-model-prevalences-with-ipsi} we display the model's
ability to capture the correlation between ipsi- and contralateral
involvement. It shows that the prevalence of metastases in the two sides
of the neck is correlated via the time of diagnosis, despite the model
not having any direct connections between the two side. However, there
are some small discrepancies in the model's prediction: When considering
the scenario of LNL II involvement in the ipsi- \emph{and} contralateral
side, it cannot quite capture the split between early and late
T-category. The model slightly overestimates the prevalence of this
scenario for early T-category patients and underestimates it for
advanced T-category.

\subsubsection{Prevalence of Midline
Extension}\label{prevalence-of-midline-extension}

\begin{figure}

\centering{

\includegraphics[width=0.5\textwidth,height=\textheight]{figures/fig-model-prevalences-midext.png}

}

\caption{\label{fig-model-prevalences-midext}Comparing the predicted
(histograms) and observed (lines depicting beta posteriors) prevalence
of midline extension for early (blue) and late (orange) T-category.
While the prevalence is predicted correctly when marginalizing over
T-category, the model cannot capture the degree of separation observed
in the data. Since the tumor's midline extension is virtually always
part of the diagnosis and hence \emph{given} when predicting a patient's
risk, we do not consider this discrepancy a major issue.}

\end{figure}%

Lastly, in figure~\ref{fig-model-prevalences-midext}, we plot the
prevalence of midline extension in the data versus our model's
prediction. It is obvious the model cannot match the large spread
between early and advanced T-category seen in the data. This is because
to achieve that, it would need to increase the advanced T-category
patient's prior distribution over diagnosis times and at the same time
reduce the probability of the tumor to cross the midline during a
time-step. But since the time-priors parameter is also coupled with the
spread probabilities among the LNLs, the model does not have that
freedom.

However, we do not consider this discrepancy a major limitation of the
model: We will not realistically be interested in the probability of
midline extension, as it is always possible to assess it with high
certainty. That is also the reason why we initially modelled the midline
extension \emph{not} as a random variable, but as a global risk factor
that would have been turned on or off from the onset of a patient's
disease evolution. This, however, lead to overly high risks for
contralateral involvement in advanced T-category patients with midline
extension, because then the model assumes an increased spread to the
contralateral side from the onset of the disease. Which is probably not
true in a majority of those cases. Thus, treating it as a random
variable that only becomes true during a patient's disease evolution
resulted in a better description of the data.

Formally, the wrong prediction of midline extension prevalence makes
little difference, since it is always given: Instead of
\(P\left( \mathbf{X}^\text{i}, \mathbf{X}^\text{c}, \epsilon \mid \mathbf{Z}^\text{i}, \mathbf{Z}^\text{c} \right)\),
we typically compute
\(P\left( \mathbf{X}^\text{i}, \mathbf{X}^\text{c} \mid \mathbf{Z}^\text{i}, \mathbf{Z}^\text{c}, \epsilon \right)\),
which does not suffer from the wrong probability of midline extension,
as the distribution over hidden states is renormalized:

\[
P \left( \mathbf{X}^\text{i}, \mathbf{X}^\text{c} \mid \mathbf{Z}^\text{i}, \mathbf{Z}^\text{c}, \epsilon \right) = \frac{P \left( \mathbf{Z}^\text{i}, \mathbf{Z}^\text{c} \mid \mathbf{X}^\text{i}, \mathbf{X}^\text{c}, \epsilon \right) P \left( \mathbf{X}^\text{i}, \mathbf{X}^\text{c}, \epsilon \right)}{P \left( \mathbf{Z}^\text{i}, \mathbf{Z}^\text{c}, \epsilon \right)}
\]

Note that a distribution over \(\epsilon\) appears both in the
enumerator and the denominator, which largely cancel each other, leaving
only the midline extension's effect on the distribution over hidden
states in the prediction.

\subsection{Prediction of Risk for Occult
Disease}\label{prediction-of-risk-for-occult-disease}

In this section, we investigate how the model may be applied clinically:
We want to estimate the risk for occult disease in some or all LNLs,
given the patient's individual diagnosis. In terms of our model, this
diagnosis consists of the T-category, the lateralization of the tumor
(does it extend over the mid-sagittal line?), and which LNLs are
clinically involved, e.g.~because some lymph nodes appear enlarged in an
MRI scan or show increased glycolytic activity on an FDG PET/CT scan.

\begin{figure}

\centering{

\includegraphics[width=0.5\textwidth,height=\textheight]{figures/fig-model-risks.png}

}

\caption{\label{fig-model-risks}Histograms over the predicted risk of
occult contralateral LNL II invovlement, shown for some combinations of
T-category, tumor lateralization, and ipsilateral clinical involvement.
The contralateral side was always assumed to be clinically negative.}

\end{figure}%

Figure~\ref{fig-model-risks} shows this predicted risk of occult disease
in contralateral LNL II for four interesting combinations of these three
risk factors. This risk is computed \emph{given} a CT diagnosis
assessing the per-LNL clinical involvement for which we assumed a
sensitivity of 81\% and a specificity of 76\%
\citep{debondt_detection_2007}.

The variable impacting the prediction for contralateral involvement in
our model is the tumor's lateralization. For example, a patient with a
clearly lateralized early T-category tumor and a clinically N0 neck is
assigned a 1-2\% risk for occult disease in contralateral LNL II. Under
the same scenario, but \emph{with} mid-sagittal extension, the risk
jumps to almost 7\%.

T-category plays a lesser role: Considering the scenario of a tumor that
crosses the mid-sagittal line and an ipsilateral neck where at least LNL
III is clinically involved, the risk for occult contralateral LNL II
disease is around 8.5\%. For the same scenario and an advanced
T-category tumor, the risk increases to 11\%.

Lastly, the predicted risk is also correlated via the time-steps to the
degree of ipsilateral involvement. Changing the aforementioned scenario
(advanced T-category, midline extension, ipsilateral LNL III clinically
involved) to one where the patient presents with a clinically N0 neck,
the risk for occult disease in the contralateral LNL II falls from 11\%
to 9.5\%.

Taken together, T-category and ipsilateral involvement may still
considerably impact the risk prediction for contralateral involvement:
In figure~\ref{fig-model-risks} the scenarios underlying the orange
(6.5\% ) and the red (11\% ) histograms differ in T-category (early vs
advanced) and clinical diagnosis (N0 vs ipsilateral at least LNL III
involved).

\section{Discussion}\label{discussion}

\begin{itemize}
\tightlist
\item
  conceptually extended model to cover both sides of neck
\item
  explicitly models midline extension (also allows for easy
  marginalization)
\item
  midline extension intuitively causes contralateral spread to become
  more similar to ipsilateral spread
\end{itemize}

\section{Acknowledgement}\label{acknowledgement}

This work was supported by:

\begin{itemize}
\tightlist
\item
  the Clinical Research Priority Program ``Artificial Intelligence in
  Oncological Imaging'' of the University of Zurich
\item
  the Swiss Cancer Research Foundation under grant number KFS
  5645-08-2022
\end{itemize}

\section{Contralateral Prevalence of
Involvement}\label{contralateral-prevalence-of-involvement}

\textsubscript{Source:
\href{https://rmnldwg.github.io/bilateral-paper/manuscript-preview.html}{Article
Notebook}}

\begin{longtable}[]{@{}llllllllllll@{}}

\caption{\label{tbl-data-strat}Contralateral involvement depending on
whether the primary tumor extends over the mid-sagittal line, the
T-category, and whether the ipsilateral LNL III was involved or
healthy.}

\tabularnewline

\caption{}\label{T_78884}\tabularnewline
\toprule\noalign{}
T-cat. & ipsi & Mid. ext. & \multicolumn{2}{l}{%
I} & \multicolumn{2}{l}{%
II} & \multicolumn{2}{l}{%
III} & \multicolumn{2}{l}{%
IV} & total \\
& & & n & \% & n & \% & n & \% & n & \% & n \\
\midrule\noalign{}
\endfirsthead
\toprule\noalign{}
T-cat. & ipsi & Mid. ext. & \multicolumn{2}{l}{%
I} & \multicolumn{2}{l}{%
II} & \multicolumn{2}{l}{%
III} & \multicolumn{2}{l}{%
IV} & total \\
& & & n & \% & n & \% & n & \% & n & \% & n \\
\midrule\noalign{}
\endhead
\bottomrule\noalign{}
\endlastfoot
early & 0 & False & 0 & 0.00 & 1 & 1.16 & 0 & 0.00 & 0 & 0.00 & 86 \\
early & 0 & True & 0 & 0.00 & 1 & 10.00 & 1 & 10.00 & 0 & 0.00 & 10 \\
early & 1 & False & 1 & 0.53 & 11 & 5.82 & 2 & 1.06 & 1 & 0.53 & 189 \\
early & 1 & True & 1 & 11.11 & 2 & 22.22 & 0 & 0.00 & 0 & 0.00 & 9 \\
early & ≥ 2 & False & 1 & 0.96 & 15 & 14.42 & 3 & 2.88 & 3 & 2.88 &
104 \\
early & ≥ 2 & True & 0 & 0.00 & 3 & 30.00 & 4 & 40.00 & 1 & 10.00 &
10 \\
advanced & 0 & False & 0 & 0.00 & 2 & 5.88 & 0 & 0.00 & 0 & 0.00 & 34 \\
advanced & 0 & True & 0 & 0.00 & 3 & 12.50 & 0 & 0.00 & 0 & 0.00 & 24 \\
advanced & 1 & False & 0 & 0.00 & 3 & 4.55 & 0 & 0.00 & 0 & 0.00 & 66 \\
advanced & 1 & True & 1 & 1.64 & 18 & 29.51 & 5 & 8.20 & 1 & 1.64 &
61 \\
advanced & ≥ 2 & False & 4 & 5.26 & 16 & 21.05 & 7 & 9.21 & 4 & 5.26 &
76 \\
advanced & ≥ 2 & True & 3 & 3.80 & 46 & 58.23 & 18 & 22.78 & 8 & 10.13 &
79 \\

\end{longtable}

\textsubscript{Source:
\href{https://rmnldwg.github.io/bilateral-paper/manuscript-preview.html}{Article
Notebook}}

\section{Marginalizing Beta Posteriors over
Unknowns}\label{marginalizing-beta-posteriors-over-unknowns}

\textsubscript{Source:
\href{https://rmnldwg.github.io/bilateral-paper/manuscript-preview.html}{Article
Notebook}}

\begin{longtable}[]{@{}llll@{}}

\caption{\label{tbl-data-unknowns}Combinations of midline extension
status and state of contralateral LNL II for early T-category patients.}

\tabularnewline

\toprule\noalign{}
Midline Extension & no & yes & unknown \\
contra LNL II & & & \\
\midrule\noalign{}
\endhead
\bottomrule\noalign{}
\endlastfoot
healthy & 352 & 23 & 59 \\
involved & 27 & 6 & 6 \\
total & 379 & 29 & 65 \\

\end{longtable}

\textsubscript{Source:
\href{https://rmnldwg.github.io/bilateral-paper/manuscript-preview.html}{Article
Notebook}}

In section~\ref{sec-prevalence} we explained that we compare the
predicted prevalence of an involvement pattern under a scenario to a
beta posterior over the ``true'' prevalence given the observations in
the data. However, our data contains incomplete observations, especially
regarding the tumor's midline extension. This information is not present
for a large group of 85 patients, most of which treated at the Centre
Léon Bérard in France.

To account for this missing information and plot the correct posteriors
over the data prevalence, we need to marginalize over the possible
combinations of matching involvement patterns and selected patients
under a scenario. For example, we have only 29 early T-category patients
with midline extension. But for an additional 65 patients we do not know
whether the tumor crossed the mid-sagittal plane. If we now want to
compute the prevalence of contralateral LNL II invovlement for these
early T-category patients with midline extension, then we need to
account for the possibility that we are missing information.

In table~\ref{tbl-data-unknowns} we show the combinations of the midline
extension status and state of the contralateral LNL II. Using only
available information, we would get to 6 out of 29 patients for the
prevalence of contralateral LNL II invovlement among the early
T-category patients with midline extension. But the column with the
``unknown'' numbers suggests that the true prevalence could range from 6
out of 29 + 59 (7\%) up to 12 out of 29 + 6 (34\%). Assuming that every
one of the possible combinations has the same a priori probability
(uniform prior) and accouting for permutations via the binomial factor,
the posterior marginalized over the unknowns could be written as

\[
\begin{aligned}
p(x; \alpha=1, \beta=1) &= \sum_{m=0}^{65} \sum_{\ell=0}^{\min(m,6)} \binom{n+m}{k+\ell} x^{k+\ell} (1-x)^{n+m - (k+\ell)} \cdot \frac{1}{\operatorname{B}(\alpha, \beta)} x^\alpha (1-x)^\beta \\
&= \sum_{m=0}^{65} \sum_{\ell=0}^{\min(m,6)} \binom{n+m}{k+\ell} x^{k+\ell+1} (1-x)^{n+m - (k+\ell) + 1}
\end{aligned}
\]

We have plotted this marginal in figure~\ref{fig-model-marginal-pdf}
(blue curve) and the terms that make up the above sum (light red
curves). Note that we weighted the plotting opacity with the number of
permutations each of the possible combinations has. The marginal is
compared against the ``naive'' beta posterior that only takes the data
into account for which midline extension is available (green curve).

\begin{figure}

\centering{

\includegraphics[width=0.5\textwidth,height=\textheight]{figures/fig-model-marginal-pdf.png}

}

\caption{\label{fig-model-marginal-pdf}Posterior distributions over the
data prevalence for contralateral LNL II involvement under the scenario
of early T-category and midline extension. The green beta distribution
shows this prevalence for those patients for which midline extension is
known. While the blue PDF is a marginal distribution summed over all
possible combinations, which are shown in shades of red. The more
permutations a particular combination has, the more opaque it is
plotted.}

\end{figure}%

\textsubscript{Source:
\href{https://rmnldwg.github.io/bilateral-paper/manuscript-preview.html}{Article
Notebook}}


  \bibliography{references.bib}


\end{document}
